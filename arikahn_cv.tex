%%%%%%%%%%%%%%%%%%%%%%%%%%%%%%%%%%%%%%%%%
% Medium Length Professional CV
% LaTeX Template
% Version 2.0 (8/5/13)
%
% This template has been downloaded from:
% http://www.Laemphplates.com
%
% Original author:
% Trey Hunner (http://www.treyhunner.com/)
%
% Important note:
% This template requires the resume.cls file to be in the same directory as the
% .tex file. The resume.cls file provides the resume style used for structuring the
% document.
%
%%%%%%%%%%%%%%%%%%%%%%%%%%%%%%%%%%%%%%%%%

%----------------------------------------------------------------------------------------
%	PACKAGES AND OTHER DOCUMENT CONFIGURATIONS
%----------------------------------------------------------------------------------------

\documentclass{resume} % Use the custom resume.cls style

\usepackage[left=0.75in,top=0.6in,right=0.75in,bottom=0.6in]{geometry} % Document margins

% \usepackage[pdftex,
%             pdfauthor={Ari E. Kahn},
%             pdftitle={Ari E. Kahn CV},
% 	    pdfsubject={Ari E. Kahn CV}]{hyperref}
% \usepackage[style=authoryear,
%             maxnames=20,
%             url=false,
%             isbn=false]{biblatex}
\usepackage[activate={true,nocompatibility},final,tracking=true,kerning=true,spacing=true,factor=1100,stretch=10,shrink=10]{microtype}
% activate={true,nocompatibility} - activate protrusion and expansion
% final - enable microtype; use "draft" to disable
% tracking=true, kerning=true, spacing=true - activate these techniques
% factor=1100 - add 10% to the protrusion amount (default is 1000)
% stretch=10, shrink=10 - reduce stretchability/shrinkability (default is 20/20)

% \addbibresource{mypublications.bib}

\name{Ari E. Kahn} % Your name

\address{210 S. 33rd St. \\ Philadelphia, PA 19104} % Your address
\address{github: ariekahn \\ linkedin: ari-kahn \\ website: www.aekahn.com}
% \address{123 Pleasant Lane \\ City, State 12345} % Your secondary addess (optional)
\address{(925)~$\cdot$~285~$\cdot$~9061 \\ arikahn@seas.upenn.edu} % Your phone number and email

\begin{document}

%----------------------------------------------------------------------------------------
%	EDUCATION SECTION
%----------------------------------------------------------------------------------------

\begin{rSection}{Education}

\textbf{University of Pennsylvania} \hfill \emph{2013--Current} \\
Ph.D. Candidate, Neuroscience \hfill Philadelphia, PA \\
Advisor: Danielle S. Bassett, Ph.D. \\
Expected Date of Completion: Spring 2019

\textbf{Washington University in St. Louis} \hfill \emph{2007--2011} \\
B.S. in Computer Science \& Chinese \hfill St. Louis, MO \\
Minor in Physics \\
Graduated with Engineering Honors, Cum Laude

\end{rSection}

%----------------------------------------------------------------------------------------
%   RESEARCH EXPERIENCE
%----------------------------------------------------------------------------------------

\begin{rSection}{Research Experience}

% \textbf{University of Pennsylvania} \hfill \emph{2013 - Current} \\
% Ph.D. Candidate, Neuroscience \hfill Philadelphia, PA \\
% Advisor: Danielle S. Bassett, Ph.D. \\

\textbf{Tel Aviv University} \hfill \emph{2012} \\
Research Assistant \hfill Tel Aviv, Israel \\
Advisor: Matti Mintz, Ph.D. \\
Computational modeling of the cerebellar microcircuit for sequential learning

\textbf{Tel Aviv University} \hfill \emph{Winter 2011} \\
Research Assistant \hfill Tel Aviv, Israel \\
Advisor: Ehud Gazit, Ph.D. \\
Implemented and refined a protocol for self-assembling nanospheres

\textbf{Technion University} \hfill \emph{Summer 2010} \\
Research Assistant \hfill Haifa, Israel \\
Advisor: Nahum Shimkin, Ph.D. \\
Implemented a machine learning based multilayer flight simulator framework

\textbf{Washington University in St. Louis} \hfill \emph{2008--2009} \\
Research Assistant \hfill St. Louis, MO \\
Advisor: William Smart, Ph.D. \\
Designed framework for BCI-based control of simulated robotic prostheses

\end{rSection}

%----------------------------------------------------------------------------------------
%   PUBLICATIONS
%----------------------------------------------------------------------------------------

\begin{rSection}{Publications}

%2019
    \textbf{Human Sensitivity to Community Structure Is Robust to Topological Variation}
    Elisabeth A. Karuza, \textbf{Ari E. Kahn}, and Danielle S. Bassett. In: Complexity, vol. 2019, Article ID 8379321 (2019).
%2018

\textbf{Network constraints on learnability of probabilistic motor sequences} \\
\textbf{Ari E. Kahn}, Elisabeth A. Karuza, Jean M. Vettel, and Danielle S. Bassett. In: Nature Human Behavior 2, pp. 936-947 (2018).

\textbf{Structure from noise: Mental errors yield abstract representations of events} \\
Chris W. Lynn, \textbf{Ari E. Kahn}, and Danielle S. Bassett. In: arXiv (2018).

%2017

\textbf{Modular Segregation of Structural Brain Networks Supports the Development of Executive Function in Youth} \\
Graham L. Baum, Rastko Ciric, David R. Roalf, Richard F. Betzel, Tyler M. Moore, Russell T. Shinohara, \textbf{Ari E. Kahn}, Simon N. Vandekar, Petra E. Rupert, Megan Quarmley, Philip A. Cook, Mark A. Elliott, Kosha Ruparel, Raquel E. Gur, Ruben C. Gur, Danielle S. Bassett, and Theodore D. Satterthwaite. In: Current Biology 27.11, p. 1561 (2017).

\textbf{Structural Pathways Supporting Swift Acquisition of New Visuomotor Skills} \\
\textbf{Ari E. Kahn}, Marcelo G. Mattar, Jean M. Vettel, Nicholas F. Wymbs, Scott T. Grafton, and Danielle S. Bassett In: Cerebral Cortex 27.1, pp. 173–184 (2017).

\textbf{Process reveals structure: How a network is traversed mediates expectations about its architecture} \\
Elisabeth A. Karuza, \textbf{Ari E. Kahn}, Sharon L. Thompson-Schill, and Danielle S. Bassett. In: Scientific Reports 7.1, p. 12733 (2017).

\textbf{Role of graph architecture in controlling dynamical networks with applications to neural systems} \\
Jason Z. Kim, Jonathan M. Soffer, \textbf{Ari E. Kahn}, Jean M. Vettel, Fabio Pasqualetti, and Danielle S. Bassett. In: Nature Physics (2017).

\textbf{Cliques and cavities in the human connectome} \\
Ann E. Sizemore, Chad Giusti, \textbf{Ari Kahn}, Jean M. Vettel, Richard F. Betzel, and Danielle S. Bassett. In: Journal of Computational Neuroscience, pp. 1–31 (2017).

\textbf{Individual Differences in Learning Social and Non-Social Network Structures} \\
Steven H. Tompson, \textbf{Ari E. Kahn}, Emily B. Falk, Jean M. Vettel, and Danielle S. Bassett. In: Journal of Experimental Psychology: Learning, Memory, and Cognition. In Press. (2018)

\textbf{Developmental increases in white matter network controllability support a growing diversity of brain dynamics} \\
Evelyn Tang, Chad Giusti, Graham L. Baum, Shi Gu, Eli Pollock, \textbf{Ari E. Kahn}, David R. Roalf, Tyler M. Moore, Kosha Ruparel, Ruben C. Gur, Raquel E. Gur, Theodore D. Satterthwaite, and Danielle S. Bassett. In: Nature Communications 8.1, p. 1252 (2017).

% 2016

% 2015

\textbf{Inter-regional ECoG correlations predicted by communication dynamics, geometry, and correlated gene expression} \\
Richard F. Betzel, John D. Medaglia, \textbf{Ari E. Kahn}, Jonathan Soffer, Daniel R. Schonhaut, and Danielle S. Bassett. In: arXiv (2015).

\textbf{Controllability of structural brain networks} \\
Shi Gu, Fabio Pasqualetti, Matthew Cieslak, Qawi K. Telesford, Alfred B. Yu, \textbf{Ari E. Kahn}, John D. Medaglia, Jean M. Vettel, Michael B. Miller, Scott T. Grafton, and Danielle S. Bassett. In: Nature Communications 6 (2015).

\end{rSection}

% \nocite{*}
% \printbibliography[heading=none, style=publisher+location+date]

%----------------------------------------------------------------------------------------
% POSTERS
%----------------------------------------------------------------------------------------
\begin{rSection}{Talks}
\textbf{Network Constraints on Learnability of Probabilistic Motor Sequences} \\
\textbf{Ari E. Kahn}, Elisabeth A. Karuza, Jean M. Vettel, Danielle S. Bassett. CompleNet. March 4--8, 2018, Boston, Massachusetts, USA.

\textbf{Network Constraints on Learnability of Probabilistic Motor Sequences} \\
\textbf{Ari E. Kahn}, Elisabeth A. Karuza, Jean M. Vettel, Danielle S. Bassett. SIAM Workshop on Network Science. July 12--13, 2018, Portland, Oregon, USA.
\end{rSection}

\begin{rSection}{Posters}
\textbf{Beyond graph topology: Walk structure influences cluster-level surprisal effects in an on-line learning task} \\
Elisabeth A. Karuza, \textbf{Ari E. Kahn}, Sharon L. Thompson-Schill, Danielle S. Bassett. Psychonomics. November 17--20, 2016, Boston, Massachusetts, USA.

\textbf{Structural Correlates of Individual Differences in Motor Sequence Learning} \\
\textbf{Ari E. Kahn}, Marcelo G. Mattar, Jean M. Vettel, Nicholas F. Wymbs, Scott T. Grafton, Danielle S. Bassett. Society for Neuroscience, November 12--16, 2016, San Deigo, California, USA.

\textbf{A model of sequential learning in the cerebellum} \\
\textbf{Ari E. Kahn}, Ari Magal, Roni Hogri and Matti Mintz.
Society for Neuroscience, October 13--17, 2012, New Orleans, Louisiana, USA.
\end{rSection}
%----------------------------------------------------------------------------------------
%   TEACHING AND MENTORING
%----------------------------------------------------------------------------------------

\begin{rSection}{Awards}
    \textbf{SIAM Student Travel Award} \hfill \emph{Spring 2018} \\
    \\
    \textbf{Jameson-Hurvich Travel Award} \hfill \emph{Fall 2016} \\
\end{rSection}

%----------------------------------------------------------------------------------------
% TEACHING AND MENTORING
%----------------------------------------------------------------------------------------

\begin{rSection}{Teaching and Mentoring}

\textbf{Teaching Assistant} \\
Introduction to Brain and Behavior \hfill \emph{Spring 2016} \\
\emph{Led weekly undergraduate recitation section and wrote testing material}
\end{rSection}

%----------------------------------------------------------------------------------------
% OUTREACH
%----------------------------------------------------------------------------------------

\begin{rSection}{Outreach}

\textbf{Upward Bound}\\
Summer Neuroscience Elective \\
Head Coordinator \hfill \emph{2016--2018} \\
Instructor \hfill \emph{2014--2015}

\textbf{Penn Neuroscience Public Lecture Series}\\
Committee Member \hfill \emph{2014--2017}

\textbf{Neuroscience Elementary School Outreach Program}\\
Instructor \hfill \emph{2013--2017}

\end{rSection}

%----------------------------------------------------------------------------------------
% PROFESSIONAL AFFILIATIONS
%----------------------------------------------------------------------------------------

\begin{rSection}{Professional Affiliations}

Society for Neuroscience

SIAM

\end{rSection}

%----------------------------------------------------------------------------------------
%	SKILLS
%----------------------------------------------------------------------------------------

\begin{rSection}{Skills}

\begin{tabular}{ @{} >{\bfseries}l @{\hspace{6ex}} l }
Programming & Python, R, Matlab, JavaScript, C, C++, LaTeX \\
Image Processing & FSL, ANTs, FreeSurfer, DTI Studio \\
% Languages & English (Native), Chinese (Intermediate), Hebrew (Intermediate)
\end{tabular}

\end{rSection}

\end{document}
