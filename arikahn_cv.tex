%%%%%%%%%%%%%%%%%%%%%%%%%%%%%%%%%%%%%%%%%
% Medium Length Professional CV
% LaTeX Template
% Version 2.0 (8/5/13)
%
% This template has been downloaded from:
% http://www.Laemphplates.com
%
% Original author:
% Trey Hunner (http://www.treyhunner.com/)
%
% Important note:
% This template requires the resume.cls file to be in the same directory as the
% .tex file. The resume.cls file provides the resume style used for structuring the
% document.
%
%%%%%%%%%%%%%%%%%%%%%%%%%%%%%%%%%%%%%%%%%

%----------------------------------------------------------------------------------------
%	PACKAGES AND OTHER DOCUMENT CONFIGURATIONS
%----------------------------------------------------------------------------------------

\documentclass{resume} % Use the custom resume.cls style

\usepackage[left=0.75in,top=0.6in,right=0.75in,bottom=0.6in]{geometry} % Document margins

% Bibliography setup
% Don't print URL, ISBN, or DOI info
% Also use full names rather than initials
\usepackage[style=authoryear,
            maxnames=30,
            url=false,
            isbn=false,
            doi=false,
            giveninits=false, sorting=ydnt]{biblatex}
% Add space between entries
\setlength\bibitemsep{0.5\baselineskip}
% Don't unindent the first line of each entry
\setlength\bibhang{0pt}
% Print Firstname-Lastname
\DeclareNameAlias{sortname}{given-family}
\addbibresource{publications_annotated.bib}

\DeclareSourcemap{
  \maps[datatype=bibtex]{
    \map{
      \step[fieldsource=eprinttype,
            match=bioRxiv,
            final]
      \step[fieldset=keywords, fieldvalue=preprint]
    }
  }
}

% Bold all highlight annotated names
\renewcommand*{\mkbibnamegiven}[1]{%
  \ifitemannotation{highlight}
    {\textbf{#1}}
    {#1}}

\renewcommand*{\mkbibnamefamily}[1]{%
  \ifitemannotation{highlight}
    {\textbf{#1}}
    {#1}}

% Microtype - better looking text
\usepackage[activate={true,nocompatibility},final,tracking=true,kerning=true,spacing=true,factor=1100,stretch=10,shrink=10]{microtype}
% activate={true,nocompatibility} - activate protrusion and expansion
% final - enable microtype; use "draft" to disable
% tracking=true, kerning=true, spacing=true - activate these techniques
% factor=1100 - add 10% to the protrusion amount (default is 1000)
% stretch=10, shrink=10 - reduce stretchability/shrinkability (default is 20/20)

\name{Ari E. Kahn} % Your name

\address{Princeton Neuroscience Institute \\ Princeton, NJ 08544} % Your address
\address{github: ariekahn \\ linkedin: ari-kahn \\ website: www.aekahn.com}
% \address{123 Pleasant Lane \\ City, State 12345} % Your secondary addess (optional)
\address{(925)~$\cdot$~285~$\cdot$~9061 \\ arik@princeton.edu} % Your phone number and email

\begin{document}

%----------------------------------------------------------------------------------------
%	EDUCATION SECTION
%----------------------------------------------------------------------------------------

\begin{rSection}{Education}

\textbf{Princeton Neuroscience Institute} \hfill \emph{2020--Current} \\
Postdoctoral Researcher \hfill Princeton, NJ \\
Advisor: Nathaniel D. Daw, Ph.D.

\textbf{University of Pennsylvania} \hfill \emph{2013--2020} \\
Ph.D. in Neuroscience \hfill Philadelphia, PA \\
Advisor: Danielle S. Bassett, Ph.D. \\
Thesis: Behavioral and Neural Correlates of Graph Learning

\textbf{Washington University in St. Louis} \hfill \emph{2007--2011} \\
B.S. in Computer Science \& Chinese \hfill St. Louis, MO \\
Minor in Physics \\
Graduated with Engineering Honors, Cum Laude

\end{rSection}

%----------------------------------------------------------------------------------------
%   RESEARCH EXPERIENCE
%----------------------------------------------------------------------------------------

\begin{rSection}{Research Experience}

% \textbf{University of Pennsylvania} \hfill \emph{2013 - Current} \\
% Ph.D. Candidate, Neuroscience \hfill Philadelphia, PA \\
% Advisor: Danielle S. Bassett, Ph.D. \\

\textbf{Tel Aviv University} \hfill \emph{2012} \\
Research Assistant \hfill Tel Aviv, Israel \\
Advisor: Matti Mintz, Ph.D. \\
Computational modeling of the cerebellar microcircuit for sequential learning

\textbf{Tel Aviv University} \hfill \emph{Winter 2011} \\
Research Assistant \hfill Tel Aviv, Israel \\
Advisor: Ehud Gazit, Ph.D. \\
Implemented and refined a protocol for self-assembling nanospheres

\textbf{Technion University} \hfill \emph{Summer 2010} \\
Research Assistant \hfill Haifa, Israel \\
Advisor: Nahum Shimkin, Ph.D. \\
Implemented a machine learning based multilayer flight simulator framework

\textbf{Washington University in St. Louis} \hfill \emph{2008--2009} \\
Research Assistant \hfill St. Louis, MO \\
Advisor: William Smart, Ph.D. \\
Designed framework for BCI-based control of simulated robotic prostheses

\end{rSection}

%----------------------------------------------------------------------------------------
%   PUBLICATIONS
%----------------------------------------------------------------------------------------

\begin{rSection}{Preprints}
% Add everything to the bibliography
\nocite{*}
% And print it without a section header
\printbibliography[heading=none, keyword=preprint]
\end{rSection}

\begin{rSection}{Publications}
\printbibliography[heading=none, notkeyword=preprint]
\end{rSection}

%----------------------------------------------------------------------------------------
% POSTERS
%----------------------------------------------------------------------------------------
\begin{rSection}{Talks}
Interdisciplinary Advances in Statistical Learning. June 27--29, 2019, San Sebastian, Spain.

CompleNet. March 4--8, 2018, Boston, Massachusetts, USA.

SIAM Workshop on Network Science. July 12--13, 2018, Portland, Oregon, USA.
\end{rSection}

\begin{rSection}{Posters}

Cognitive Computataional Neuroscience. August 6--9, 2024, Cambridge, Massachusetts, USA.
 
Cognitive Computataional Neuroscience. August 25--28, 2022, San Francisco, California, USA.

PNI Retreat. May 2--3, 2022, Vernon Township, New Jersey, USA.

Interdisciplinary Advances in Statistical Learning. June 27--29, 2019, San Sebastian, Spain.

Sackler Colloquium ``Brain Produces Mind by Modeling''. May 1--3, 2019, Irvine, California, USA.

MINS Symposium. April 3, 2019, Philadelphia, Pennsylvania, USA.

Conference on Computataional Neuroscience. September 5--8, 2018, Philadelphia, Pennsylvania, USA.

Psychonomics. November 17--20, 2016, Boston, Massachusetts, USA.

Society for Neuroscience. November 12--16, 2016, San Deigo, California, USA.

Society for Neuroscience. October 13--17, 2012, New Orleans, Louisiana, USA.
\end{rSection}
%----------------------------------------------------------------------------------------
%   AWARDS
%----------------------------------------------------------------------------------------

\begin{rSection}{Awards}
    \textbf{Sackler Colloquium ``Brain Produces Mind by Modeling'' Travel Award} \hfill \emph{Spring 2019} \\
    \\
    \textbf{SIAM Student Travel Award} \hfill \emph{Spring 2018} \\
    \\
    \textbf{Jameson-Hurvich Travel Award} \hfill \emph{Fall 2016} \\
\end{rSection}

%----------------------------------------------------------------------------------------
% TEACHING AND MENTORING
%----------------------------------------------------------------------------------------

\begin{rSection}{Teaching and Mentoring}

\textbf{Teaching Assistant} \\
Introduction to Brain and Behavior \hfill \emph{Spring 2016} \\
\emph{Led weekly undergraduate recitation section and wrote testing material}

\textbf{Undergraduate Mentoring} \\
Erin Chang \hfill \emph{2024 -- current} \\
\emph{Supervising senior thesis, neural data analysis}

Nathaniel Nyema \hfill \emph{2018 -- 2020} \\
\emph{Mentoring on behavioral experiments and data analysis}
\end{rSection}

%----------------------------------------------------------------------------------------
% OUTREACH
%----------------------------------------------------------------------------------------

\begin{rSection}{Outreach}

\textbf{Upward Bound}\\
Summer Neuroscience Elective \\
Head Coordinator \hfill \emph{2016--2018} \\
Instructor \hfill \emph{2014--2015}

\textbf{Penn Neuroscience Public Lecture Series}\\
Committee Member \hfill \emph{2014--2017}

\textbf{Neuroscience Elementary School Outreach Program}\\
Instructor \hfill \emph{2013--2017}

\end{rSection}

%----------------------------------------------------------------------------------------
% PROFESSIONAL AFFILIATIONS
%----------------------------------------------------------------------------------------

\begin{rSection}{Professional Affiliations}

Society for Neuroscience

SIAM

\end{rSection}

%----------------------------------------------------------------------------------------
%	SKILLS
%----------------------------------------------------------------------------------------

\begin{rSection}{Skills}

\begin{tabular}{ @{} >{\bfseries}l @{\hspace{6ex}} l }
Programming & Python, R, Matlab, JavaScript, C, C++, Julia, LaTeX \\
Image Processing & FSL, ANTs, FreeSurfer, DSI Studio, Nipype \\
% Languages & English (Native), Chinese (Intermediate), Hebrew (Intermediate)
\end{tabular}

\end{rSection}

\end{document}
